\documentclass[11pt]{scrartcl}
\usepackage[utf8]{inputenc}
\usepackage[spanish]{babel}

\title{\textbf{Legal Aspects}}
\subtitle{Roles and Organizational Policies}
\author{Ricardo García Fernández}
\date{\today}

\begin{document}

\maketitle

\section{Exercise}

\textbf{Please, read GPL version 2, section 8, and check if this section would be in conflict with any criterion of the Open Source Definition (OSD). If you think so, which OSD point would fail? Why? What do you think it is OSD-compliant despite that?}\\

\subsection{GPLv2 section 8}

\emph{8. If the distribution and/or use of the Program is restricted in certain countries either by patents or by copyrighted interfaces, the original copyright holder who places the Program under this License may add an explicit geographical distribution limitation excluding those countries, so that distribution is permitted only in or among countries not thus excluded. In such case, this License incorporates the limitation as if written in the body of this License.}\\

GPLv2 \textbf{Section 8} means that if in some countries due to a patent violation, or interfaces could not distribute the software could add an extra clause to prevent the distribution license in the relevant countries to stay that way 'free'. So would violate the free distribution with respect to the 'OSD' for everyone to access without restriction or discrimination described in clause 5:\\

\emph{5. No Discrimination Against Persons or Groups\\
The license must not discriminate against any person or group of persons.}\\

So, "\emph{What do you think it is \textbf{OSD-compliant} despite that?}". Because, after both licenses reread several times I think it is 'OSD-compliant' if we claim the section 7 "Distribution of License":

\emph{7. Distribution of License\\
The rights attached to the program must apply to all to whom the program is redistributed without the need for execution of an additional license by those parties.}

Which clearly shows that the use of this license are allowed the rights described in it regardless of any law enforcement outside, so there would be no problem of distribution in certain countries concerning \emph{Section 8 GPL Version 2}.

\subsection{GPL Version 3}

In \emph{GPL Version 3} this problem disappears because the license as it is described in section 11:

\emph{If you convey a covered work, knowingly relying on a patent license, and the Corresponding Source of the work is not available for anyone to copy, free of charge and under the terms of this License, through a publicly available network server or other readily accessible means, then you must either (1) cause the Corresponding Source to be so available, or (2) arrange to deprive yourself of the benefit of the patent license for this particular work, or (3) arrange, in a manner consistent with the requirements of this License, to extend the patent license to downstream recipients. “Knowingly relying” means you have actual knowledge that, but for the patent license, your conveying the covered work in a country, or your recipient's use of the covered work in a country, would infringe one or more identifiable patents in that country that you have reason to believe are valid.}

The licensor is obliged to allow the use of patents worldwide. That is, has to get patents are not infringed in any country having licensed the software with this license.
This license removes the software user responsibilities not violating any rules because the licensor is obliged to provide free access to all the code under this license.

\begin{thebibliography}{9}

\bibitem{gplv2}
    GPL Version 2,\\
    http://www.gnu.org/licenses/gpl-2.0.html

\bibitem{osd}
    Open Source Definition,\\
    http://opensource.org/osd    

\end{thebibliography}

\end{document}