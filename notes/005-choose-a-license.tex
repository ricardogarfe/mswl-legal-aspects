\documentclass{scrartcl}
\usepackage[utf8]{inputenc}
\usepackage[spanish]{babel}

\title{\textbf{Lesson 5}}
\subtitle{Choose a License}
\author{Miguel Vidal}
\date{\today}

\begin{document}

\maketitle

\section{Previous criteria}

Hay unos pasos previos a tener en cuenta a la hora de elegir una licencia:
\begin{itemize}
    \item Tipos de licencia.
    \item Cuando.
    \item Libertades básicas, es decir uso de licencias libres.
    \item Mayor grado de diseminación, licencias permisivas.
    \item Mantener el control sobre la evolución del software; pactos de reciprocidad, copyleft.
    \item Reconocimiento de la autoría, patentes, etc.
\end{itemize}
Se han de seguir estos pasos previamente para la elección de la licencia, no a la inversa ya que nos podemos encontrar con reestricciones con las que no contábamos por el mero hecho de haber elegido la licencia \emph{por simpatía} y no por estos criterios descritos.

\section{Evaluación de la Licencia}
\emph{Comprobar en FSF, OSI, Debian project, Wikipedia (EN)\footnote{Wikipedia no es una fuente fiable sin que se comprueben las referencias}, analogies/templates, Check 4 freedoms/OSD}.\\

Este proceso se redacta para no tener que llegar a leer la licencia, es decir seguir unas guías donde se publican las licencias de software libre\footnote{Ver gráfico de las diapositivas de clase} y comprobar la existencia de la misma evitando toparnos con ella.

\section{Casos}

\begin{itemize}
    \item Obras derivadas.
    \item Autoría ? 
    \item Enlaces a mi trabajo. Copyleft reducido, débil.
    \item Evitar restricciones. Estables estándar, permisivo.
    \item Debe ejecutarse con uno en particular. GPL.
    \item Patentes en mi programa ? Permisivas, copyleft.
\end{itemize}

\section{Aplicar la licencia}

\begin{itemize}
    \item LICENSE o COPYING archivo.
    \item Copyright y resumen de la licencia en cada archivo de código fuente.
    \item Al menos 
    \item Información sobre cómo contactar con los licenciatarios
    \item Si es interactivo, mostrar la licencia al iniciarlo o a través del comando de ayuda.
    \item Si tiene una GUI se puede incluir en el menú la información sobre la licencia.
\end{itemize}

Método archivo por archivo; con un resumen en la cabecera de cada archivo. Según la FSF está más sujeto a errores debido a los posibles cambios.
Método único; Un fichero donde se muestra la licencia. Puede acabar un fichero sin licencia por lo tanto, todos los derechos reservados y no sería libre.

\section{Dual Licencing}

El porqué del doble licenciamiento, posibles:
\begin{itemize}
    \item Distribución de un software bajo dos o más licencias (Perl, Mozilla/Firefox, MySQL).
    \item Segregación de mercados; basado en modelos de negocio (MySQL enterprise).
    \item Ofrecer versiones previas a la liberación.
\end{itemize}

Evita incompatibilidades mediante la creación de \emph{n} versiones por cada licencia definida en el proyecto.
Eben Moglen; creó Sotfware Freedom Law Center\footnote{http://www.softwarefreedom.org/}. Según esta organización (entre BSD y GPL) es mejor utilizar BSD para que así no se pierda la licencia del fichero (librería, etc...) en si al incorporarlo dentro de un proyecto GPL.

\section{Incompatibilidad}

No cumplan todos los términos de ambas licencias, es decir que en una en relación a la otra tienen incompatibilidades. Mientras \textbf{no se haga una distribución} no se valoran las licencias.

\begin{itemize}
    \item Copyleft; suelen ser incompatibles a menos que se declare explícitamente con cual es compatible.
    \item Linked; 
    \item Bifurcación (fork); Nombre diferente y otra licencia. GPL mantiene la licencia en cada fork. BSD no es necesario, es decir depende del licenciador.
    \item En Software Privativo no puede producirse como sabemos.
    \item GPL tiene tendencia a evitar la bifurcación ya que obliga a mantener la licencia.
    \item Las licencias tipo BSD son propensas al fork.
\end{itemize}

Existen casos de éxito a partir de un fork, por ejemplo; openBSD/386BSD, Gnu-Emacs/Xemacs, GForge/SourceForge, OpenSolaris/Illumos.

\end{document}