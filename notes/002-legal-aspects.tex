\documentclass[11pt]{scrartcl}

\title{\textbf{Legal Aspects\\
				Lesson 2}}
\subtitle{Copyright on Software}
\author{Miguel Vidal}
\date{\today}

\usepackage{amsmath}
\begin{document}

\maketitle

\section{Introducci\'on}

Cuando y como empez\'o ?
\begin{itemize}
	\item Proceso antimonopolio en contra de \textbf{IBM} que dur\'o 13 a\~nos.
	\item Impidieron que AT\&T fuera l\'ider en software y hardware m\'ovil dejando el c\'odigo de unix a la universidad de Berkeley para que m\'as adelante sacaran la primera versi\'on \textbf{BSD}.
	\item Portabilidad del \emph{sistema operativo} y \textbf{C} permiti\'o que el software no estuviera ligado intr\'isecamente al hardware.
	\item Bifurcaci\'on de la que nace la industria del software.
	\item A partir de estos eventos tenemos un resultado inesperado: unix y la industria del software.
\end{itemize}

\subsection{Razones}

Copyright en vez de patentes para el software.
\begin{itemize}
	\item Autom\'atico
	\item Protegida por copyright
	\item No tiene costes
	\item Documentaci\'on bajo copyright
	\item Internacionalizado
	\item Harmoniza con otros trabajos.
\end{itemize}
El porqu\'e, \emph{scope}:
\begin{itemize}
	\item C\'odigo fuente y binario.
	\item Descripci\'on del programa.
	\item Material adicional.
	\item Interfaces.
	\item Bases de Datos.
\end{itemize}
No entra en el copyright:
\begin{itemize}
	\item algoritmos
	\item procesos
	\item t\'ecnicas ytilizadas en el desarrollo
\end{itemize}

\section{Licencias}

Por que necesitamos una licencia ?\\
Si \textbf{no} licencias tu c\'odigo, tu c\'odigo \textbf{no} puede ser usado \textbf{por nadie m\'as}.\\
Opci\'on de no usar licencias, es decir, dominio p\'ublico.\\
Contrato \emph{sinalagm\'atico} (unilateral).\\
No se han de firmar las condiciones a diferencia de \textbf{EULA}.

\subsection{Dominio P\'ublico}

Donde van las obras cuando expira el \emph{copyright}.\\
Conflictos en Espa\'na con la donaci\'on al dominio p\'ublico de una obra:
\begin{itemize}
	\item Derechos morales, retirada de la obra.
	\item Derecho de copia privada.
\end{itemize}
Soluci\'on de \emph{creative commons}.\\
La licencia \textbf{CC0}\footnote{http://creativecommons.org/publicdomain/zero/1.0/} para facilitar la publicaci\'on en el domino p\'ublico.

\subsection{FLOSS}

Si no se usa licencia, todos los derechos reservados.\\
\textbf{Legal Hacking}: define los derechos base del autor a trav\'es del copyright.\\
M\'axima: \emph{FLOSS are consistnet with IP laws}.

\subsubsection{Tipos de licencias FLOSS}

Debe cumplir las cuatro libertades:
\begin{enumerate}
	\item uso
	\item copia
	\item edici\'on
	\item redistribuci\'on
\end{enumerate}


\subsubsection{Academicas}

\begin{itemize}
	\item Las m\'as simples con muy pocas reestricciones.
	\item Cualquier tipo de uso incluso generando un software privativo.
	\item Provienen de las universidades.
	\item MIT, BSD, ISC
\end{itemize}	

\subsubsection{Permisivas}

\begin{itemize}
	\item Similares a las acad\'emicas.
	\item Marcas registradas, protecci\'on de la marca.
	\item Cualquier tipo de uso incluso generando un software privativo.
	\item Concesi\'on de patentes.
	\item Ejmplo: Licencia Apache; Android.
	\item Anti patentes ya que \emph{obliga} a que se permita utilizar patentes si alguien las incluye permitiendo que los usuarios puedan utilizar las patentes adjuntas.
\end{itemize}

\subsubsection{Parcialmente cerrables}

\begin{itemize}
	\item Protege el 'fichero' para que su licencia no cambie y no obliga a que otros componentes adopten su licencia.
	\item BSD al enlazar y al modificar copyleft.
	\item MPL, CDDL, LGPL; 
\end{itemize}

\subsubsection{Rec\'iprocas, copyleft}

Copyleft \emph{fuerte}.
\begin{itemize}
	\item V\'iricas, se reproduce cuando se utiliza con otro c\'odigo.
	\item Mant\'en siempre la misma licencia.
\end{itemize}

\subsection{Copyleft \& Permisivas}

Todas son software libre pero englobadas en estos dos grandes grupos:
\begin{enumerate}
	\item Copyleft: Pueden hacer obras derivadas privativas.
	\item Permisivas: No se pueden hacer obras derivadas privativas.
\end{enumerate}

\subsubsection{Permisivas}

Restricciones permisivas.
\begin{itemize}
	\item Atribuci\'on a los autores.
	\item Transmisi\'on de libertades.
	\item Mantener las libertades.	
	\item Protecci\'on de libertades. Antipatentes, DRM.
\end{itemize}

\subsection{Compatibilidad}

Compatibilidad entre las licencias.

\subsection{Dual-lisencing}

Diferentes t\'erminos de licencias.

\subsection{Proliferaci\'on de Licencias}

Incrementaron exponencialemnte y complic\'o las compatibilidades entre distintas distribuciones.

\subsection{Patentes}

Duraci\'on de 20 a\'nos para pasar al dominio p\'ublico. Para ser una patente ha de cumplir que sea \'util, novedos, no tenga arte previo.
\begin{thebibliography}{9}

	\bibitem{microsoft}
		CNNExpansion,\\
		\emph{UE: Microsoft viola normas antimonopolio}.\\
		http://www.cnnexpansion.com/negocios/2009/01/16/ue-microsoft-viola-normas-antimonopolio
\end{thebibliography}
\end{document}