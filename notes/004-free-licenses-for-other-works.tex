\documentclass{scrartcl}

\title{\textbf{Lesson 4}}
\subtitle{Free Licenses for other intellectual works}
\author{Miguel Vidal}
\date{\today}

\begin{document}

\maketitle

\section{Introducci\'on}

De donde vienen ? El concepto de cultura libre nace a partir de la definici\'on de software libre.\\

Ejemplos; audio, pel\'iciulas, v\'ideos, obras cient\'ificas, filos\'oficas, etc... obras originales por el simple hecho de ser creadas.\\
A trav\'es de Stallman el cual hace una distinci\'on entre dos tipos de obras:
\begin{itemize}
    \item obras funcionales; documentaci\'on, enciclopedias, manuales...
    \item obras no funcionales; literatura, m\'usica, pel\'iculas...
\end{itemize}

\subsection{FSF: Opiniones}

Propone obras privativas.\\
Recomienda la \textbf{by-nd} de creative-commons\footnote{http://creativecommons.org/choose/}.\\
Verbatim Copying Licenses\footnote{http://www.gnu.org/licenses/}.

No se pueden hacer traducciones a no ser que el propietario de los derechos te de permiso, por lo tanto no es cultura libre o software libre.

\subsection{Freedom Definied}

Freedom Defined Initiative by Mako Hill. Consigui\'o el consenso para la aplicaci\'on de las libertades del software a otro tipo de obras mediando entre FSF, CC y Wikimedia.

\begin{itemize}
    \item Libertad para usar.
    \item Estudiar.
    \item Cambios y mejoras.
    \item Distribuir obras derivadas.
\end{itemize}

Condiciones adicionales a\~nadiendo unas restricciones:
\begin{itemize}
    \item Failitar el c\'odigo fuente.
    \item Formato libre sin patentes.
    \item Sin restricciones t\'ecnicas (no DRM).
    \item Nada que impida las 4 libertades.
\end{itemize}

Copyleft; cuatro libertades y mantener la licencia.

Non-Commercial Share-alike \textbf{No} es copyleft.

\section{GNU FDL}

FSF licencia para obras \emph{no funcionales}. Debian: Todas las obras bajo esta licencia no son libres (citaci\'on). Se resolvi\'o eliminando la documentaci\'on que contuviese partes destacadas como "invariantes".

\section{Wikipedia}

Hasta Junio 2009 estaba cubierta por la GFDL y pas\'o a licenciarse con \emph{CC-by-sa 3.0}.
No estaba preparada para los recursos online.
No era compatible con \emph{CC-by-sa} ya que ambas son copyleft.

\section{Documentaci\'on permisiba}

\subsection{BSD}

FreeBSD tiene su propia licencia de documentaci\'on que aplica a su documentaci\'on. Es parecida a la licencia BSD de dos cl\'ausulas.

La de tres cl\'ausulas.

\subsection{ISC License}

Parecida la BSD de dos cl\'ausulas pero m\'as legible, sin hablar ni de c\'odigo fuente ni binario.

\section{Creative-Commons}

Dos caminos posibles dentro de creative-commons:
\begin{itemize}
    \item Donar al dominio p\'ublico.

    \item Cuatro reglas
    \begin{itemize}
        \item Atribuci\'on
        \item Non-commercial.
        \item No derivado
        \item Share alike, no es combinable con \emph{no derivado}
    \end{itemize}
\end{itemize}

\subsection{Non-commercial}

Es la cl\'ausula que m\'as quebraderos de cabeza genera:
\begin{itemize}
    \item no est\'a claro el uso comercial directo o indirecto.
    \item incertidumbre sobre lo que es una actividad comenrcial.
    \item Incompatibilidad con proyectos libres como Wikipedia.
    \item Confusi\'on alrededor de la palabra "libre".
\end{itemize}

\begin{thebibliography}{9}

\bibitem{spring-inside}
  SpringSource Community,\\
  http://www.springsource.org/

\end{thebibliography}

\end{document}