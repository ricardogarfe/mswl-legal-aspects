\documentclass[11pt]{scrartcl}

\title{\textbf{Legal Aspects\\
				Lesson 3}}
\subtitle{FLOSS Licensing}
\author{Miguel Vidal}
\date{\today}

\begin{document}

\maketitle

\section{Introducci\'on}

Si no licencias tu c\'odigo no se puede compartir.\\
Existen m\'as de 1000 licencias.\\
Licencias recomendadas:
\begin{itemize}

	\item Acad\'emica/permisivas puede utilizarse todo tipo de software.
	\item Weak Copyleft, mantener la misma licencia.
	\item Strong Copyleft, Obligan a ser copyleft.
\end{itemize}

\section{Academic}

\subsection{BSD Licenses}

BSD es una variante de Unix que se remonta a los a\~nos 80 (1988). Exist\'ia previamente a GPL.

\begin{itemize}

	\item Es conocida como Minimalista.
	\item Redistribuir el c\'odigo sin a\~nadir ninguna restricci\'on.
	\item Actual licencia \textbf{BSD} se basa en la descrita en el a\~no 1988.
	\item No existe garant\'ia.
\end{itemize}

Funcionalmente es como el dominio p\'ublico siempre y cuando contemos con el c\'odigo fuente disponible.

\subsubsection{Ventajas}

No hay restricciones a la evoluci\'on del software como por ejemplo la estandarizaci\'on de un producto.

\begin{itemize}

	\item Integraci\'on en un proyecto privativo.
	\item Puede convertirse en c\'odigo abierto o privado.
	\item No existe complicaci\'on legal como en \emph{GPL} o \emph{LGPL}. No existen violaciones de la licencia\footnote{}.
	\item No hace falta invertir mucho tiempo adaptarla y no da quebraderos de cabeza.
\end{itemize}

\subsubsection{4 Cl\'ausulas}

A partir de la base \emph{Redistribuci\'on y uso del siguiente c\'odigo.}:

TODO: Completar con las 4 cl\'ausulas.

Redistributions of source code must retain the above copyright notice this list of conditions adn the following disclaimer.

\subsubsection{3 Cl\'ausulas}

La cl\'ausula de publicidad desapareci\'o debido a que se sobrecargaba por la referencia al creador y arrastraba todas las anteriores debido a la creaci\'on desmedida de las licencias utilizando esta como plantilla.

\subsubsection{2 Cl\'ausulas}

Desaparece la \'ultima.

\subsubsection{Disclaimer}

Nada de garant\'ia a no ser que se proporcione un servicio asociado al software.

\subsection{BSD-like}

\subsubsection{ISC}

La licencia m\'as corta de FLOSS que enfatiza con claridad en referencia a los permisos; \emph{usar, copiar, modificar y/o distribuir}.\\
Ejemplos; openBSD...

\subsubsection{MIT}

Es m\'as expl\'icita referente a los derechos que otorga \emph{uso, copia, modificaci\'on, pubicaci\'on... y/o vender copias}.\\
Ejemplos; Putty, RubyOnRails, X11...

\subsubsection{Zope}

\subsubsection{WTFPL}

"Do What The Fuck You Want To Public License".

\subsubsection{Ejemplos}

Darren Reed a trav\'es de ipfilter ya que no se pod\'ia modificar el c\'odigo del programa por lo tanto \textbf{no} es software libre.
Theo de Raadt propuso eliminarlo de la distribuci\'on de BSD y encontrar un sustituto "we will have to work on an alternative".\\
Se cre\'o \emph{Packet Filter} que se consolid\'o como la alternativa, pfsense dentro de FreeBSD pero no existe versi\'on para Linux.

\section{Permisiva}

\subsection{Apache}

Originalmente era una BSD acad\'emica de 3 cl\'ausulas.
Puede ser integrada en proyectos de software privado.
\emph{Apache License 2.0} (Enero de 2004):
\begin{itemize}
	\item Facilitar el uso en proyectos generales fuera de la Apache Software Foundation (ASF).
	\item Facilitar el uso de patentes a todo el mundo, es decir no hacer pagar por ello al que lo usa con una licencia. Cada contribuidaor se beneficiar\'a de una licencia de uso
	\item Patent retaliation: intenta litigar con la excusa de que hay c\'odigo patentado pierde el derecho a utilizarlo.
\end{itemize}

Compatible con GPLv3 pero s\'olo convirti\'endose a GPLv3 e incompatible con GPLv2.

\section{Weak copyleft}

S\'olo se preocupa de su propio fichero fuente, mantiene el fichero fuente con la misma licencia que se ha definido.\\

MPL 1.1:
\begin{itemize}
	\item cl\'ausula anti patentes.
\end{itemize}
Ejemplos; MPL (Mozilla Public License).Firefox se acoge a m\'ultiples licencias (MPL, GPL, LGPL).

MPL 2.0:
\begin{itemize}
	\item Compatible con Apache y GPL.
\end{itemize}

\subsection{CDDL}

Basada en MPL como obra derivada que se cre\'o por Sun Microsystems para el proyecto Solaris:
\begin{itemize}

	\item OSI-Compliant 2004.
	\item Incompatible con GPL.
\end{itemize}

\subsection{LGPL}

Lesser GPL, originalmente significaba \emph{Library} para poder estandarizar las librer\'ias del proyecto para que de esta forma no se convierta todo en copyleft siendo as\'i una permisiba:
\begin{itemize}

	\item Cl\'ausula para enlazar con cualquier librer\'ia. Linking exception; una obra que usa una librer\'ia.
\end{itemize}
Ejemplos; OpenOffice, etc...\\
\'Esta es la \'unica diferencia con la GPL.\\

\section{Strong Copyleft}

\subsection{GPL}

GNU Public License.
\begin{itemize}
	\item primera licencia que implement\'o el concepto de \emph{copyleft}. Todas las licencias copyleft que se han creado han partido de \'esta.
	\item Impedir la comercializaci\'on privativa del software.
	\item No se puede combinar con otro c\'odigo con una licencia que no sea compatible.
\end{itemize}

\subsubsection{Copyleft}

Modificar copia o copias del programa siempre que:
\begin{itemize}
	\item Avisar que se ha cambiado y cuando
	\item Distribuir bajo esta misma licencia
\end{itemize}

\subsubsection{No EULA}

No es necesario la aceptaci\'on esta licencia mediante una firma.

\subsubsection{Versions}

Historia de las distintas versiones GPL:
\begin{itemize}
	\item GPL version 1 (1989): "Version 1 or later." Gen\'erica independiente del programa.
	\item GPL version 2 (1991): "Libery or dead" por 'culpa' de las patentes.
	\item GPL version 3 (2007): Patentes, DRM, Tivoizaci\'on a trav\'es de la comunidad.
\end{itemize}

\subsubsection{Affero AGPL}

Esta licencia se define con una cl\'ausula extra que obliga a distribuir el c\'odigo cuando el software es SaaS.

\subsubsection{Interpretaci\'on FSF}

\begin{itemize}
	\item Enlazar un programa GPL en la compilaci\'on produce un programa GPL.
	\item La salida de un programa \emph{no} es un programa derivado.
	\item Linux es GPL, cualquier c\'odigo enlazado debe ser GPL. Drivers por separado; NVidia, etc...
\end{itemize}

\subsubsection{Interpretaci\'on de Lawrence Rosen}

Lawrence Rosen\footnote{http://en.wikipedia.org/wiki/Lawrence\_Rosen\_(attorney)} Interpreta que:

Si existe un enlace din\'amico a una librer\'ia GPL el software resultante no es GPL ya que no es una obra derivada sino un obra colectiva.
Es la aplicaci\'on del copyright estadounidense la que corresponde con la obra derivada de la FSF crea una incongruencia.

\section{Conclusi\'on}

perspectivas y necesidades en un determinado momento.

\begin{thebibliography}{9}

\bibitem{licenciasosi}
  OSI Licences.

\end{thebibliography}
\end{document}
