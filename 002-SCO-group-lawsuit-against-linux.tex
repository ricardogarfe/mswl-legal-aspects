\documentclass[11pt]{scrartcl}

\title{\textbf{Legal Aspects}}
\subtitle{Intellectual Property: The SCO Group lawsuit against Linux}
\author{Miguel Vidal}
\date{\today}

\begin{document}

\maketitle

\section{Execise}

In 2003 SCO Group claimed a cluster of legal complaints about Unix and Linux source code:
 
\begin{enumerate}

	\item What alleged IP violations SCO claimed? ¿What companies was involved?

		\begin{itemize}
			\item IBM wrote and donated to be incorporated into Linux was added in violation of SCO's contractual rights.
			\item \emph{SCO Group filed suit against IBM for \$1 Billion (later raised to \$3 Billion) on unsubstantiated charges of having transferred SCO intellectual property to the Linux community in violation of IBM's license for AIX Unix, through which IBM had access to SCO's source code. SCO claims attempts to negotiate with IBM were fruitless. IBM says they were never approached at all on the matter.}\footnote{http://www.aaxnet.com/editor/edit032.html\#ibm}.
			
			\item as a conlcusion, SCO claimed to be the owner of Unix copyrigths and intellectual property of Unix therefore Linux that is Unix heritage breaks intellectual property rights using IBM's donations from Unix into Linux distributions.

			\item \emph{IBM, Novell and Red Hat claimed against SCO} to defend its rigths and the users. FSF is not a company...
		\end{itemize}
		
	\item How the SCO litigation was resolved? What was the final verdict?

	\emph{Novell, not the SCO Group, is the rightful owner of the copyrights covering the Unix operating system. The court also ruled that "SCO is obligated to recognize Novell's waiver of SCO's claims against IBM and Sequent}\footnote{http://www.groklaw.net/article.php?story=20081120195227418}.\\
	SCO was not the owner of Unix copyrights so this company has no rights to do nothing agains the users and the other companies involved. Novell was declared as the official owner of intellectual property and didn't want to purse Unix copyrights\footnote{http://www.pcworld.com/article/135959/article.html}.

	\item Now, search the Web and read "Open Letter"\footnote{http://www.sco.com/copyright/} from Darl McBride, CEO of SCO, dated December 4, 2003:	

	\begin{enumerate}
		\item What does the letter argue about copyleft scheme and GPL license?
			\begin{itemize}
				\item \emph{increasingly rancorous legal controversy over violations of our UNIX intellectual property contract, and what we assert is the widespread presence of our copyrighted UNIX code in Linux}
				\item \emph{U.S. copyright law versus the GNU GPL}.
				\item patent laws
				\item "the motive of profit is the engine that ensures the progress of science."
				
				\item \textbf{Answer}: the main argue about copyleft and GPL license is that USA allways claims about copyright for cience improvements and using this kind of 'rights' coul be the worst decision ever and will make the country lots of damages to innovation and investigation. Defines GPL and copyleft as evil, that wants to steal the ideas of the people and taking profit without any consideration of the investigator that is bare in front of this model without paterns. Paterns is the word that is more used to defend copyright in this letter, as the old fashioned way to defend the creativity but paterns how we saw in different examples [footnote] aren't needed or aren't useful in software development, because of the use and because could dead before they will be released to public domain and become useful.
				
			\end{itemize}
		\item Do you think GPL and free software licenses was threatened in any way? Why? (or why not?)
		
		Yes, 
	\end{enumerate}

\end{enumerate}


\begin{thebibliography}{9}

	\bibitem{ilaw}
		iLaw,\\
		SCO v IBM – Lessons for Developers,\\
		http://www.ilaw.com.au/public/scoarticle.html

	\bibitem{axxnet}
		Axxnet,\\
		SCO - Death Without Dignity,\\
		http://www.aaxnet.com/editor/edit032.html
	\bibitem{controversies}
	  Wikipedia,\\
	  SCO–Linux controversies.\\
	  http://en.wikipedia.org/wiki/SCO-Linux\_controversies.

	\bibitem{timeline}
		Wikipedia,\\
		Timeline of SCO–Linux controversies,\\
		http://en.wikipedia.org/wiki/Timeline\_of\_SCO-Linux\_controversies.

\end{thebibliography}
\end{document}
